\documentclass[a4paper, 18pt]{article}

% Faire des marges un peu moins large que celles par défaut
\usepackage[top=20mm, bottom=20mm, left=18mm, right=18mm]{geometry}
\usepackage{ucs}
\usepackage[utf8x]{inputenc} % Pour l'encodage 
% Reconnaitre les caratères accentués dans le source.
\usepackage[T1]{fontenc} 
% Meilleurs polices
%\usepackage{concmath}
\usepackage{lmodern}
\usepackage[francais]{babel}
% Insertion d'images
\usepackage{graphicx}

\title{PAQL du PLD SPIE}

\begin{document}

\part{Objet et caractéristiques du plan d'assurance et contrôle qualité :}

\section{Objectifs du plan}

Ce projet porte sur la refonte du système d'information d'une entreprise existante, SPIE, afin de mieux répondre à leurs exigences. Cela va impliquer une étude approfondie de l'existant et des besoins de l'entreprise, ainsi qu'une évaluation des différentes solutions techniques qui s'offrent à nous, à savoir l'utilisation d'un progiciel existant, ou alors le développement d'une solution spécifique.

Un premier objectif du plan d'assurance qualité est d'assurer le bon déroulement ainsi que la cohérence de notre démarche de travail.

Le projet va nous amèner à générer quantité importante de livrables, certains à l'attention du client et d'autres nécéssaires à assurer la cohérence de notre prestation, dont nous nous devons d'assurer la qualité. Le second objectif de ce plan sera donc de spécifier précisement les méthodes que nous prévoyons d'appliquer pour garantir la qualité de l'ensemble de nos livrables.

\section{Domaine d'Application}

Ce plan s'applique à l'ensemble du projet. Néanmoins, afin d'en simplifier la compréhension, on peut distinguer trois sous-domaines d'application :

\paragraph{Niveau organisationnel :} Afin de garantir l'application des processus d'assurance qualité au sein du groupe de travail, il est nécéssaire de mettre en place des règles et consignes précises définissant l'organisation de l'équipe.

\paragraph{Niveau production :} Définition des rôles de chacun et des méthodes et "bonnes pratiques" à appliquer au quotidien lors de la production de documents relatifs au projet.

\paragraph{Niveau outils :} Outils nous permettant de faciliter l'application de notre processus de qualité. Nous allons aussi définir les différentes pratiques s'appliquant à ses outils pour garantir l'éfficacité de leur usage.

\section{Responsabilités au sein de l'équipe}

L'équipe de projet a pour rôle la réalisation du plan d'assurance qualité, ainsi le suivi de son application tout au long du projet.

\paragraph{Chef de projet : Mathieu Coquet} Le rôle du CdP est de valider la PAQ, de l'appliquer et de le faire respecter. C'est lui qui arrête les décisions et qui fait la validation chaque document avant livraison.

\paragraph{Résponsables qualité : Quentin Calvez, Jan Keromnes} Le rôle des RQ est de rédiger et d'améliorer le PAQ, ainsi que de garantir son application.

\paragraph{Experts, Consultants} Leur rôle est d'appliquer le PAQ, et d'apporter les corrections nécessaires lors d'un écart pour revenir vers une conformité avec la PAQ.

\section{Documents Applicables et Documents de références}

\subsection{Documents Livrables}

Liste des documents livrables du projet. Pourra être amenée à changer légérement au cours de celui-ci.

\paragraph*{Phase d'initialisation}

\begin{itemize}
\item Dossier d’initialisation
\item Plan d’assurance qualité
\end{itemize}


\paragraph*{Phase d'expression des besoins}
\begin{itemize}

\item Etude de l'existant
\item Rapport de synthèse et de benchmarking
\item Dossier d’expression des besoins
\item Scénarios envisagés :

\begin{itemize}
\item Rapport solution 1 : Développement spécifique
\item Rapport solution 2 : ERP 
\end{itemize}

\item Dossier de modélisation et de configuration standard

\end{itemize}


\paragraph*{Evaluation}
\begin{itemize}
\item Dossier de choix 
\end{itemize}

\paragraph*{Restitiution} 
\begin{itemize}
\item Dossier Bilan
\end{itemize}

\section{Critères et procédures d'évolution du PAQ}

Description des critères d'évolution du PAQ, et des procédures à suivre dans le but de faire évoluer ce dernier en prévenant l'entrée de "mauvaises" pratiques.

\section{Critères et procédures de dérogation au PAQ}

Description des cas "limités" de dérogation au PAQ, et des conséquences liées.

%\part{Système qualité mis en œuvre pour le projet :}
%
%\section{Objectifs et Engagements qualité pour le projet}
%
%Descriptions des objectifs du SQ mis en action pour le projet.
%
%\section{Mesure de la Qualité}
%
%Description des moyens de mesure (quantification) de la qualité dans le cadre de l'étude, et du projet.
%
%\section{Documentation Qualtité du projet}
%
%Description des différents documents de mise en place et de suivi du système qualité
%
%\section{Activités d'assurance et de contrôle de la qualité }
%
%Description des actions d'application du SQ / PAQ menées au sein du projet 

% Il faudra voir où on met ce truc.
\section{Activités d'assurance et de contrôle de la qualité}

\subsection{Modèles et guides et style}

Chaque projet doit spécifier, lors de son initialisation, le modèles et le guide de style que tout document atenant au projet devra respecter.
Ces guides servent de cadre à la rédaction de ces documents. Ils s'assurent de leur facilité de lecture, mais aussi de maintenance par les différents membres de l'équipe qui auront à collaborer dessus.

\subsection{Processus de suivi de documentation}

En plus de devoir respecter le guide de style établi, chaque librable devra faire l'objet d'un ticket Redmine permettant de fédérer les informations sur ce livrable, et de suivre sont état d'avancement. Le devra comporter les informations suivantes :

\begin{itemize}

\item Auteur du document
\item Responsable de la validation du document
\item Date d’initialisation et de dernière mise à jour
\item Date de livraison
\item État du document

\end{itemize}

\part{Conduite du projet}

\section{Organisation du Projet}

TODO.

\section{Outils de mesure de la Qualité mis en oeuvre pour le projet}

Pour contrôler la qualité des documents produits, nous utilisons des moyens de mesure pour quantifier la qualité dans le cadre de ce projet. L'évaluation de la qualité de nos livrables d'effectuera de la manière suivante :

\paragraph*{Qualité syntaxique / orthographique :} Importance de la qualité rédactionnelle et de la forme des documents.
\paragraph*{Qualité et pertinence du contenu par rapport à la démarche globale :} Nous nous devons d’assurer une cohérence entre nos livrables avec un accent sur le fond des documents.
\paragraph*{Tracabilité du document :} Afin de faciliter l’établissement d’un suivi qualité et d’un état des lieux du projet.
\paragraph*{Respect du guide de style instauré au sein du groupe de travail}

\section{Documentation qualité du projet}

\section{Planification et suivi du projet }

La planification prévisionnelle a été effectuée dans le dossier d'initialisation.

Au fil de l'avancement, les tâches sont affectées aux membres de l'équipe projet grâce à l'outil de gestion de projet Redmine, et un email est envoyé aux personnes concernées. Le suivi des tâches est intégré dans Redmine, et il est de la responsabilité des membres de l'équipe de mettre à jour leurs tâches (temps passé, avancement, remarques, etc.) sur l'outil. Un accès pourra être fourni à la MOA dans un but de transparence et de facilité de suivi.

La mesure de l'avancement se fait d'une part par les experts techniques, qui indiquent leur avancement sur la plateforme de gestion de projet, et d'autre part par le responsable qualité, qui pourra juger de l'avancement des revues de documents.

Les réunions de projet se font une fois par séance de travail, et un compte rendu consultable par la MOA sera placée sur le Wiki de l'outil de gestion de projet. À la fin de chaque réunion, la prochaine est planifiée. Les compte rendu de réunion sont rédigé de manière à résumer tout ce qu'il s'est dit pendant la réunion, sans synthèse, de manière à refléter de manière fidèle le dialogue de l'équipe.

En ce qui concerne le suivi prévisionnel, le chef de projet dresse après chaque session de travail un indicateur de l'état du projet, visible sur le tableau de bord de suivi du projet. L'outil de gestion de projet permet aussi d'avoir des statistiques globales sur le projet, incluant le nombre d'heure passé par acteur sur chaque catégorie de tâche, permettant d'avoir une vue d'ensemble du projet.

\part{Gestion de la documentation}

\section{Identification de la documentation}

Dans le cadre de ce projet, nous mettons en place une politique de référencement de la documentation. Chaque document livrable est identifié par un nom unique (voir liste des livrables). De plus, dans un but de contrôle et de traçabilité, nous utiliserons le logiciel de gestion de version Git, permettant de conserver l'historique de toutes les modifications apportées sur chaque document, et si besoin de pouvoir revenir en arrière dans le temps sur tout ou partie d'un document. Pour garantir le suivi de l'évolution des documents, la bonne façon de faire est de procéder par petites modifications incrémentales facilement identifiables. L'état d'un document dans le temps est alors identifiable par un numéro de révision associé au dernier changement effectué. Ainsi, il est possible d'identifier un document par son nom et son état dans le temps de manière précise et systématique.

\section{Sauvegarde et Archivage}

Afin d'assurer une traçabilité constante, toutes les modifications sont conservées par le système de versionnement Git. De plus, afin d'éviter toute perte de données, l'information est dupliquée: en effet, Git permet à chaque membre du projet d'avoir son propre clone du repository central, et de le synchroniser au fur et à mesure de son avancement. Ceci permet, en cas de panne majeur, de ne perdre aucune donnée. Ainsi, le travail sur tous les documents est systématiquement sauvegardé et archivé, assurant le contrôle de l'évolution des documents.

\part{Contrôle des fournisseurs}

\section{Documents de liaison}

Listings des documents et présentation de ces derniers, concernant la liaison avec les fournisseurs (contrats, closes de partenariat...)

\part{Terminologie}

\section{Lexique}

\begin{itemize}
\item Git: Logiciel de gestion de version décentralisé qui permet de stocker un ensemble de fichiers en conservant la chronologie de toutes les modifications qui ont été effectuées dessus.
\end{itemize}

\section{Abréviations}

\begin{itemize}
\item CdP: Chef de Projet
\item RQ: Responsable Qualité
\item SPIE: Entreprise concernée par la présente étude de cas
\item PAQ: Plan d'Assurance Qualité
\item MoA: Maîtrise d'Ouvrage
\item MoE: Maîtrise d'Oeuvre
\item SI: Système d'Information
\end{itemize}

\end{document}
