\documentclass[a4paper, 18pt]{article}

% Faire des marges un peu moins large que celles par défaut
\usepackage[top=20mm, bottom=20mm, left=18mm, right=18mm]{geometry}
\usepackage{ucs}
\usepackage[utf8x]{inputenc} % Pour l'encodage 
% Reconnaitre les caratères accentués dans le source.
\usepackage[T1]{fontenc} 
% Meilleurs polices
%\usepackage{concmath}
\usepackage{lmodern}
\usepackage[francais]{babel}
% Insertion d'images
\usepackage{graphicx}

\title{Initialisation du projet longue durée GSTP}
\author{JLevesy}


\begin{document}

\part{Objet et caractéristiques du plan d'assurance et contrôle qualité :}

\section{Objectifs du plan}


Nous sommes ici dans la situation d'un bureau d'études 

\section{Domaine d'Application}

Description des domaines d'action du Plan d'assurance qualité

\section{Responsabilités de réalisation et suivi du plan}

Répartition des responsabilités de réalisation au sein du groupe de travail

\section{Documents Applicables et Documents de références }

Description de la documentation 

\subsection{Documents Applicables}

Documents générés par le groupe de travail.

\subsection{Documents de Référence}

Documents utilisés par le groupe de travail.

\section{Critères et procédures d'évolution du PAQ}

Description des critères d'évolution du PAQ, et des procédures à suivre dans le but de faire évoluer ce dernier en prévenant l'entrée de "mauvaises" pratiques.

\section{Critères et procédures de dérogation au PAQ}

Description des cas "limités" de dérogation au PAQ, et des conséquences liées.

\part{Système qualité mis en œuvre pour le projet :}

\section{Objectifs et Engagements qualité pour le projet}

Descriptions des objectifs du SQ mis en action pour le projet.

\section{Mesure de la Qualité}

Description des moyens de mesure (quantification) de la qualité dans le cadre de l'étude, et du projet.

\section{Documentation Qualtité du projet}

Description des différents documents de mise en place et de suivi du système qualité

\section{Activités d'assurance et de contrôle de la qualité }

Description des actions d'application du SQ / PAQ menées au sein du projet 

\part{Conduite du projet : }

\section{Organisation du Projet }

\section{Présentation des activités couvertes par le projet }

\section{Planification et suivi du projet }


\part{Gestion de la documentation :}

\section{Identification de la documentation}

Description de la politique de référencement de la documentation appliquée dans le cadre du projet. 

\section{Sauvegarde et Archivage}

Description de la politique de Sauvegarde, et d'archivage de la documentation.

\part{Contrôle des fournisseurs :}

\section{Documents de liaison}

Listings des documents et présentation de ces derniers, concernant la liaison avec les fournisseurs (contrats, closes de partenariat...)

\part{Terminologie :}

\section{Lexique}

\section{Abréviations} 

\end{document}