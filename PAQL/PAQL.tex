\documentclass[a4paper, 18pt]{article}

% Faire des marges un peu moins large que celles par défaut
\usepackage[top=20mm, bottom=20mm, left=18mm, right=18mm]{geometry}
\usepackage{ucs}
\usepackage[utf8x]{inputenc} % Pour l'encodage 
% Reconnaitre les caratères accentués dans le source.
\usepackage[T1]{fontenc} 
% Meilleurs polices
%\usepackage{concmath}
\usepackage{lmodern}
\usepackage[francais]{babel}
% Insertion d'images
\usepackage{graphicx}

\title{PAQL du PLD SPIE}

\begin{document}

\part{Objet et caractéristiques du plan d'assurance et contrôle qualité :}

\section{Objectifs du plan}

Ce projet porte sur la refonte du système d'information d'une entreprise existante, SPIE, afin de mieux répondre à leurs exigences. Cela va impliquer une étude approfondie de l'existant et des besoins de l'entreprise, ainsi qu'une évaluation des différentes solutions techniques qui s'offrent à nous, à savoir l'utilisation d'un progiciel existant, ou alors le développement d'une solution spécifique.

Un premier objectif du plan d'assurance qualité est d'assurer le bon déroulement ainsi que la cohérence de notre démarche de travail.

Le projet va nous amèner à générer quantité importante de livrables, certains à l'attention du client et d'autres nécéssaires à assurer la cohérence de notre prestation, dont nous nous devons d'assurer la qualité. Le second objectif de ce plan sera donc de spécifier précisement les méthodes que nous prévoyons d'appliquer pour garantir la qualité de l'ensemble de nos livrables.

\section{Domaine d'Application}



\section{Responsabilités de réalisation et suivi du plan}

Répartition des responsabilités de réalisation au sein du groupe de travail

\section{Documents Applicables et Documents de références }

Description de la documentation 

\subsection{Documents Applicables}

Documents générés par le groupe de travail.

\subsection{Documents de Référence}

Documents utilisés par le groupe de travail.

\section{Critères et procédures d'évolution du PAQ}

Description des critères d'évolution du PAQ, et des procédures à suivre dans le but de faire évoluer ce dernier en prévenant l'entrée de "mauvaises" pratiques.

\section{Critères et procédures de dérogation au PAQ}

Description des cas "limités" de dérogation au PAQ, et des conséquences liées.

\part{Système qualité mis en œuvre pour le projet :}

\section{Objectifs et Engagements qualité pour le projet}

Descriptions des objectifs du SQ mis en action pour le projet.

\section{Mesure de la Qualité}

Description des moyens de mesure (quantification) de la qualité dans le cadre de l'étude, et du projet.

\section{Documentation Qualtité du projet}

Description des différents documents de mise en place et de suivi du système qualité

\section{Activités d'assurance et de contrôle de la qualité }

Description des actions d'application du SQ / PAQ menées au sein du projet 

\part{Conduite du projet : }

\section{Organisation du Projet }

\section{Présentation des activités couvertes par le projet }

\section{Planification et suivi du projet }

\part{Gestion de la documentation :}

\section{Identification de la documentation}

Dans le cadre de ce projet, nous mettons en place une politique de référencement de la documentation. Chaque document
livrable est identifié par un nom unique (voir liste des livrables). De plus, dans un but de contrôle
et de traçabilité, nous utiliserons le logiciel de gestion de version Git, permettant de conserver l'historique de toutes
les modifications apportées sur chaque document, et si besoin de pouvoir revenir en arrière dans le temps sur tout ou
partie d'un document. Pour garantir le suivi de l'évolution des documents, la bonne façon de faire est de procéder par petites
modifications incrémentales facilement identifiables. L'état d'un document dans le temps est alors identifiable par un
numéro de révision associé au dernier changement effectué. Ainsi, il est possible d'identifier un document par son nom
et son état dans le temps de manière précise et systématique.

\section{Sauvegarde et Archivage}

Afin d'assurer une traçabilité constante, toutes les modifications sont conservées par le système de versionnement
Git. De plus, afin d'éviter toute perte de données, l'information est dupliquée: en effet, Git permet à chaque membre
du projet d'avoir son propre clone du repository central, et de le synchroniser au fur et à mesure de son avancement.
Ceci permet, en cas de panne majeur, de ne perdre aucune donnée. Ainsi, le travail sur tous les documents est
systématiquement sauvegardé et archivé, assurant le contrôle de l'évolution des documents.

\part{Contrôle des fournisseurs :}

\section{Documents de liaison}

Listings des documents et présentation de ces derniers, concernant la liaison avec les fournisseurs (contrats, closes de partenariat...)

\part{Terminologie :}

\section{Lexique}

\section{Abréviations} 

\end{document}
